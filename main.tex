\documentclass[oneside, 11pt]{book}
% article, IEEEtran, proc, minimal, report, book, slides, memoir, letter, beamer

%\usepackage[utf8]{inputenc}
\usepackage[english]{babel}
\usepackage{amsmath,amsfonts,amsthm,amssymb}
\usepackage[dvipsnames]{xcolor}
\usepackage{soul}
\usepackage{graphicx}
\usepackage{pdflscape}
\usepackage{float}
\usepackage{booktabs}
\usepackage[labelfont=bf]{caption}
\usepackage{listings}
\usepackage[toc]{appendix}
\usepackage{tocbibind} % includes the table of contents in the table of contents
% \usepackage[nottoc]{tocbibind} % does not include the table of contents in the table of contents

% from the CVPR style
\usepackage{eso-pic}
\usepackage{xspace}

\usepackage{subcaption} % to allow subfigures within a larger float
\usepackage{placeins} % for float barrier

\usepackage{multirow} % table formatting

\usepackage{xcolor} % for highlighting changes

\usepackage{hanging}
\usepackage{longtable}

\setcounter{secnumdepth}{4} % make subsubsections have numbers

% code listing formatting
\lstset{
  %numbers=left,
  frame=single,
  %frame=top,frame=bottom,
  %captionpos=b,
  %xleftmargin=17pt,
  framexleftmargin=17pt,
  framexrightmargin=17pt,
  framexbottommargin=5pt,
  %framextopmargin=5pt,
  basicstyle=\linespread{1.0}\ttfamily\small,
  escapeinside={(*}{*)},
  breaklines=true,
  breakindent=0.1em,
  % postbreak=\raisebox{0ex}[0ex][0ex]{\ensuremath{\color{red}\hookrightarrow\space}} % If you want line breaks in code to have red arrows or not
}


\usepackage{setspace}
%\onehalfspacing
\doublespacing

\usepackage[breaklinks,hidelinks]{hyperref}

\usepackage[margin=1in]{geometry}

\pagestyle{plain}

\renewcommand{\contentsname}{Table of Contents}
\renewcommand\lstlistlistingname{List of Code Listings}

\renewcommand{\lstlistoflistings}{\begingroup
\tocfile{\lstlistlistingname}{lol}
\endgroup}

\newcommand\hlt[1]{\textcolor{red}{#1}} % custom command to turn text red to mark changes

% Definition of macros (from CVPR template)
\makeatletter
\DeclareRobustCommand\onedot{\futurelet\@let@token\@onedot}
\def\@onedot{\ifx\@let@token.\else.\null\fi\xspace}
\def\etal{\emph{et al}\onedot}
\def\eg{\emph{e.g}\onedot}
\def\ie{\emph{i.e}\onedot}
\def\Eg{\emph{E.g}\onedot}
\makeatother

\lstnewenvironment{todo}[1][]{
  \lstset{ %
    basicstyle=\ttfamily#1\setstretch{1},
    backgroundcolor=\color{orange},
    postbreak=,
  }
}{}

\newcommand{\tabitem}{~~\llap{\textbullet}~~}

\begin{document}

\frontmatter

\begin{center}
  {\scshape\Huge [A Title That is Clever and to the Point] \par}
  \vspace{1cm}
	{\Large [Author Name] \par}
	\vspace{1.5cm}
	{\large A dissertation submitted to the Faculty of Graduate Studies\\
    in partial fulfilment of the requirements \\
    for the degree of\par}
	\vspace{0.8cm}
  {\scshape\Large Doctor of Philosophy \par}
	\vspace{1cm}
	{\large Graduate Program in\\ Electrical Engineering and Computer Science\\
    York University\\
    Toronto, Ontario, Canada\par}
  \vspace{0.5cm}
  {TBD-Month, TBD-Year \par}
  \vfill
  {\textcopyright [Author Name], TBD \par}
\end{center}

\thispagestyle{empty}

\input{sec/front/abstract.tex}
\input{sec/front/acknowledgements.tex}
\tableofcontents
\listoftables
\listoffigures
\lstlistoflistings
\chapter{List of Abbreviations}

\begin{longtable}[H]{p{.20\textwidth}  p{.80\textwidth} }

AAE & An Abbreviation Example \\
% & \\
% &
\end{longtable}


\mainmatter

\chapter{Introduction}
\label{ch:Intro}

\newpage

%!TEX root = ../../main.tex

% This contains the main introduction paragraphs which sets everything up in the broadest of terms;
% no section headings are included here, since the text goes directly within the Chapter heading of intro

Let's get this dissertation party started!


\chapter{Content Chapter}
\label{ch:SomeLabel}

  % statement of contribution and publications, if applicable
  \singlespace
  \emph{The work in this chapter has been published previously as the following:}

  \vspace{4mm}
  \hangpara{5mm}{1}
    Dissertation Author, Author 2, Supervisor ``My awesome paper'', in \emph{Journal on my topic}, [Year]
  \vspace{4mm}

  \hangpara{-5mm}{-1}
  \emph{and has been updated and extended here for the purposes of this document. Author 2 did [explain Author 2's contributions].}
  \newpage
  \doublespace

% Then whatever input sections go here
% \input{sec/chapter_name/chapter_text.tex}

% Repeat for as many content chapters as you have

\chapter{Conclusions and Future Directions}
\label{ch:Conclusion}

\newpage

%!TEX root = ../../main.tex

% This section contains the overall conclusions and future work

\section{Summary of Contributions}

I did stuff.

\section{Future Directions}

I will do stuff.


\bibliography{bibliography}{}
\bibliographystyle{ieeetr} % use whatever style your field prefers

%\begin{appendices}
%\chapter{Additional Implementation Details}\label{ch:appendix}
%
%\input{sec/appendix/appendixA.tex}
%\end{appendices}

\end{document}
